\documentclass[article,11pt,letterpaper,fleqn]{article}
\usepackage{graphicx,color}
\usepackage{array}
\usepackage{threeparttable}
\usepackage[format=hang,font=normalsize,labelfont=bf]{caption}
\usepackage{colortbl}
\usepackage{multirow}
\usepackage{geometry}
\usepackage{subfigure}
\geometry{letterpaper,tmargin=1in,bmargin=1in,lmargin=1.25in,rmargin=1.25in}
\usepackage{hyperref}
\hypersetup{colorlinks,%
citecolor=red,%
filecolor=red,%
linkcolor=red,%
urlcolor=blue,%
pdftex}
\usepackage{amsmath}
\usepackage{amssymb}
\usepackage{amsthm}
\usepackage{harvard}
\usepackage{tikz}
\usepackage{setspace}
\usepackage{float,graphicx,color}
\usepackage{appendix}
\usepackage{longtable}
\newtheorem*{thm}{Theorem}
\theoremstyle{definition}
\usepackage{lscape}
\numberwithin{equation}{section}
\newcommand{\cn}{\citeasnoun} % shortens command to cite as noun
\newcommand\ve{\varepsilon}


\title{Guide to the Marginal Effective Tax Rate Calculationsl}
\date{\today}



% make tables with smaller sized font 
\makeatletter
\def\table{\@ifnextchar[{\table@i}{\table@i[\fps@table]}}
\def\table@i[#1]{\@float{table}[#1]\footnotesize}
\makeatother



%\setlength{\topmargin}{-0.4in}
%\setlength{\topskip}{0.3in}    % between header and text
%\setlength{\textheight}{9.0in} % height of main text
%\setlength{\textwidth}{6in}    % width of text
%\setlength{\oddsidemargin}{39pt} %even side margin
%\setlength{\evensidemargin}{39pt} %odd side margin

\begin{document}
\bibliographystyle{aer}
\maketitle



\begin{abstract}
This guide outlines the equations and data used to computer marginal effective tax rates on new investment by industry and tax treatment.
\end{abstract}

\section{Introduction}

The B-Tax model produces estimates of the marginal effective tax rates on new investment under the baseline tax policy and user-specified tax reforms.  These effective rate calculations take two forms.  The \emph{marginal effective tax rate} (METR) provides the tax wedge on new investment at the level of the business entity.  The \emph{marginal effective total tax rate} (METTR) includes individual level taxes in the measure of the tax wedge on new investment.  One can think of the former as indicating the effect of taxes on incentives to invest from the perspective of the firm and the latter as representing effect of taxes on incentives to invest from the perspective of the saver.

As \cn{FullertonMETR} notes, calculations of METRs depend on several assumptions.  These include those relating to equilibrium in capital markets. discount rates, inflation rates, investor expectations, churning, who investments are financed, how risk is treated, and whether one believes the ``old view" or ``new view" of dividend taxes better represents investment incentives.  Our equilibrium assumptions include the assumption that the marginal investment earns an after-tax rate of return equal to the market rate of return, returns across asset types are equalized, investors' risk-adjusted returns from debt and equity are equalized.  Real discount rats and inflation rates are taken from the Congressional Budget Offices forecasts of nominal interest rates and inflation (\textcolor{red}{unclear if we should use the series or just have constant rates}).  Regarding risk and expectations, we assume no uncertainty in investment returns.  We use historical data on the time equities are held and how investment is financed to inform effective tax rates on capital gains and financial policy decisions, respectively.  We use the ``old view" of dividend taxation in our calculations of the METTRs, implying that dividend taxes affect investment incentives.  We further describe the implications of these assumptions in the relevant sections below.

The methodology to calculate METRs and METTRs follows closely \cn{CBO_ETRs}.

One should note that these METR and METTR calculations include only federal tax policy (current law or the user specified proposal) and therefore exclude the effects of state and local tax policy on investment incentives.  Integrating such policies is a worthwhile project, but the results here are generally sufficient for comparing alternative federal tax proposals.


This guide is organized as follows...

\section{Marginal Effective Tax Rates}

The marginal effective tax rate is calculated as the expected pre-tax rate of return on a marginal investment minus the real after-tax rate of return to the business entity, divided by the pre-tax rate of return on the marginal investment.  That is: 

\begin{equation}
METR_{i,m,j} = \frac{\rho_{i,m,j} - (r_{m,j}-\pi)}{\rho_{i,m,j}},
\end{equation}

\noindent\noindent where the subscripts $i$, $m$, and $j$ refer to the type of asset, the production industry, and the tax entity type (e.g., C-corporation, partnership, S-corporation).  The parameter $\rho_{i,m,j}$ is the pre-tax rate of return on the marginal investment, $r_{m,j}$ is the business entity's nominal after-tax rate of return and $\pi$ is the rate of inflation (so that $r_{m,j}-\pi$ is the real after-tax rate of return).  It is assumed that the business entity discounts future cash flow by the rate $r_{m,j}$.  By definition, the marginal investment is the investment whose before tax return is equivalent to the cost of capital, $\rho_{i,m,j}$.  The cost of capital is given by:

\begin{equation}
\rho_{i,m,j} = \frac{(r_{m,j}-\pi+\delta_{i})}{1-u_{j}}(1-u_{j}z_{i})+w_{i,m,j}-\delta_{i},
\end{equation}  

\noindent\noindent where $\delta_{i}$ is the rate of economic depreciation, $u_{j}$ is the statutory income tax rate at the first level of taxation (e.g., at the business entity level for C-corporations and at the individual level for pass-through business entities), $z_{i}$ is the net present value of deprecation deductions from a dollar of new investment, and $w_{i,m,j}$ is the property tax rate.  

We calculate the cost of capital, $\rho_{i,m,j}$, separately for each type of asset, production industry, and tax treatment (corporate or non-corporate).  See the Table \ref{tab:param_list} for the degree to which the various parameters vary across these three dimensions.  

% Table generated by Excel2LaTeX from sheet 'All'
\begin{landscape}
\begin{table}[htbp]
  \centering
  \caption{METR Calculation Parameters to Calibrate}
    \begin{tabular}{llllll}
    \hline
    \hline
    Parameter & Description & Source & Vary by asset & Vary by industry & Vary by tax treatment \\
    \hline
    $\pi$ & Inflation rate & CBO Economic Forecast, user input & No    & No    & No \\
    $i$   & Nominal interest rate & CBO Economic Forecast, user input & No    & No    & No \\
    $E_{c}$ & Required real return on corporate equity & Historical rates - take CBO? & No    & No    & N/A - only for corp \\
    $u_{j}$ & Statutory business entity-level income tax & Internal Revenue Code, user input & No    & No    & Yes \\
    $w_{i,m,j}$ & Property tax rate & User input - zero at federal level under current law & Maybe & Maybe & Maybe \\
    $z_{i}(y)$ & Tax depreciation allowance & Internal Revenue Code, user input & Yes   & No    & No \\
    $\delta_{i}$ & Economic depreciation rate & Bureau of Economic Analysis & Yes   & No    & No \\
    $f_{m,j}$ & Fraction of investment financed with debt & Financial Accounts of U.S. + SOI data & No    & Yes   & Yes \\
          &       &       &       &       &  \\
    $\alpha_{d,ft,j}$ & Fraction of debt held in fully taxable accounts & Financial Accounts of U.S. + SOI data & No    & No    & Yes \\
    $\alpha_{d,td,j}$ & Fraction of debt held in tax deferred accounts & Financial Accounts of U.S. + SOI data & No    & No    & Yes \\
    $\alpha_{d,nt,j}$ & Fraction of debt held in non-taxable account & Financial Accounts of U.S. + SOI data & No    & No    & Yes \\
    $\alpha_{e,ft,j}$ & Fraction of equity held in fully taxable accounts & Financial Accounts of U.S. + SOI data & No    & No    & Yes \\
    $\alpha_{e,td,j}$ & Fraction of equity held in tax deferred accounts & Financial Accounts of U.S. + SOI data & No    & No    & Yes \\
    $\alpha_{e,nt,j}$ & Fraction of equity held in non-taxable account & Financial Accounts of U.S. + SOI data & No    & No    & Yes \\
    $m_{c}$** & Fraction of earnings retained by corporations & Financial Accounts of U.S. + SOI data & No    & No    & N/A - only for corp \\
    $\tau_{div,j}$ & Dividend tax rate on marginal investor & Tax Calculator (?) & No    & No    & Yes \\
    $\tau_{int,j}$ & Interest tax rate on marginal investor & Tax Calculator (?) & No    & No    & Yes \\
    $Y_{scg}$ & Number of years hold stock subject to short term gain & SOI   & No    & No    & N/A - only for corp \\
    $Y_{lcg}$ & Number of years hold a stock subject to long term gains & SOI   & No    & No    & N/A - only for corp \\
    $\tau_{scg}$ & Short term capital gains rate on marginal investor & Tax Calculator (?) & No    & No    & N/A - only for corp \\
    $\tau_{lcg}$ & Long term capital gains rate on marginal investor & Tax Calculator (?) & No    & No    & N/A - only for corp \\
    $\omega_{scg}$ & Share of taxable equity investments held for & SOI   & No    & No    & N/A - only for corp \\
    & less than one year  & & & & \\
    $\omega_{lcg}$ & Share of taxable equity investments held for  & SOI   & No    & No    & N/A - only for corp \\
    & more than one year, but not until death & & & & \\
    $\omega_{xcg}$ & Share of taxable equity investments held until death & SOI   & No    & No    & N/A - only for corp \\
    $\tau_{td,j}$ & Tax on deferred capital income for marginal investor & Tax Calculator (?) & No    & No    & Yes \\
    $Y_{td,j}$ & Number of years hold a tax deferred investment & Take CBO's number - 8 years & No    & No    & Yes \\
          &       &       &       &       &  \\
    & ** I'm using the CBO's notation for $m$ here.&  We should probably change it since & we are &  using  $m$ & to represent industry.        \\
    \hline
    \hline
    \end{tabular}%
  \label{tab:param_list}%
\end{table}%
\end{landscape}

At times users may be interested in the variation in $METR$s across asset types, in which case we can use the $METR$ calculation outlined above. At other times users may wish to view the variation in $METR$s across industry.  In this case, we compute a weighted average cost of capital for each production industry and tax treatment as follows:

\begin{equation}
\rho_{m,j} = \frac{\sum_{i=1}^{I}\widetilde{FA}_{i,m,j}\rho_{i,j}}{\sum_{i=1}^{I}\widetilde{FA}_{i,m,j}} ,
\end{equation}

\noindent\noindent where the subscripts $i$, $m$, and $c\in\{C,NC\}$, refer to the asset type, production industry, and tax treatment.  The calculation of the variable $\widetilde{FA}_{i,m,c}$ is discussed below.

With the cost of capital for all fixed assets in an industry-tax treatment grouping, we then compute the $METR$ of the industry and tax treatment as:

\begin{equation}
METR_{m,j} =  \frac{\rho_{m,j} - (r_{m,j}-\pi)}{\rho_{m,j}},
\end{equation}

\subsection{Nominal Discount Rates}

The nominal discount rate, $r_{m,j}$, used by the business represents the cost of funds to the business.  These funds may come from equity, either through retained earnings or new equity issues, or from debt.  The cost of equity is given by $E_{j}$ (and varies by tax treatment), the cost of debt is given by the nominal interest rate $i$ (and is the same for all businesses).  The variable $E_{j}$ represents the expected real rate of return that investors can expect if they invest in any business of entity type $j$.  In general, interest payment deductions may be deductible, thus the cost of debt is only $i(1-u_{j})$, where $u_{j}$ is the statutory tax rate on business income at the first level (\textcolor{red}{We should, at some point, set up the equation to allow for ACE type of systems, also allow for haircut to interest deduction}).  We assume that the cost of funds for the marginal investment is a weighted average of the cost of funds from these two sources, debt and equity.  In particular:

\begin{equation}
r_{m,j}-\pi = f_{m,j}\left[i(1-u_{j})-\pi\right] + (1-f_{m,j})E_{j},
\end{equation}

\noindent\noindent where $f_{m,j}$ represents the fraction of the marginal investment financed with debt by firms in industry $m$ and of tax entity type $j$.

\subsection{NPV of Depreciation Deductions}

The net present value of depreciation deductions is solved for using the discount rate derived above.  Specifically, we have: 

\begin{equation}
z_{i} = \int_{0}^{Y}z_{i}(y)e^{-r_{m,j}y}dy,
\end{equation}

\noindent\noindent where $Y$ is the number of years the asset is depreciated over, $y$ is time in years, $z_{i}(y)$ is the dollar value of deprecation deductions in year $y$ per dollar invested in asset of type $i$, and $e$ is the mathematical constant.  The function $z_{i}(y)$ reflects tax policy regarding deprecation schedules.  

\subsection{Parameters}

In order to calculate $METR$s, we need to assign values to each of the parameters described above.  

This section will go through the derivation of some of the parameters above, in particular the $E_{j}$.


\section{METRs for Inventories and Land}

Two classes of assets, inventories and land, necessitate slightly modifications from the above methodology when computing $METR$s.  This section discusses those modifications.

\subsection{Inventories}

Need to talk about inventory accounting methods....

\subsection{Land}



\section{METRs for Owner-Occupied Housing}

\section{Marginal Effective Total Tax Rates}

$METTR$s include taxation at all levels, at the business entity and the individual to whom the returns from investment ultimately accrue.  Note that when there is no entity level tax (as is the case with non-C-corporate entities under current law), then the $METTR$ is equal to the $METR$.  The $METTR$ is computed as:

\begin{equation}
METTR = \frac{\rho_{i,m,j}-s_{m,j}}{\rho_{i,m,j}}
\end{equation}

\noindent\noindent In equation above, $s_{m,j}$ is the overall after-tax return to savers from an investment in a business entity operating in production industry $m$ and organized as a entity of type $j$ .  We compute this return as:

\begin{equation}
s_{m,j} = f_{m,j}s_{d,m,j} + (1-f_{m,j})s_{e,m,j},
\end{equation}

\noindent\noindent where $f_{m,j}$ is the fraction of the investment that is financed with debt (and corresponds to the same fraction used in the calculation of the cost of capital noted above) and $s_{d,m,j}$ and $s_{e,m,j}$ are the after-tax returns to the saver from debt and equity, respectively.  These in turn are found as:

\begin{equation}
s_{d,m,j} = \alpha_{d,ft,j}\times \left[i(1-\tau_{int}-\pi\right] + \alpha_{e,td,j}\times s_{d,td,j} + \alpha_{d,nt,j}\times (i-\pi)
\end{equation}

Here, $\alpha_{d,ft,j}$, $\alpha_{d,td,j}$, and $\alpha_{d,nt,j}$ are the fraction of debt of entities of tax treatment $j$ held in fully taxable, tax deferred, and non-taxable accounts.  The variable $s_{d,td,j}$ are the after-tax returns of tax-deferred debt investors in entities of type $j$.  The tax rate on interest income is $\tau_{int}$ and the nominal interest rate and inflation are given by $i$ and $\pi$.


 The return on tax deferred accounts is:
 
 \begin{equation}
s_{d,td,j} = \frac{1}{Y_{td,j}}ln \left[(1-\tau_{td,j})e^{iY_{td,j}}+\tau_{td,j}\right]-\pi
\end{equation}


The after-tax return on equity investments are given by:

\begin{equation}
s_{e,j} = \alpha_{e,ft,j}\times s_{e,ft,j} + \alpha_{e,td,j}\times s_{e,td,j} + \alpha_{e,nt,j}\times E_{j}
\end{equation}

 

Here, $\alpha_{e,ft,j}$, $\alpha_{e,td,j}$, and $\alpha_{e,nt,j}$ are the fraction of equity held in fully taxable, tax deferred, and non-taxable accounts.  The variables, $s_{e,ft,j}$ and $s_{e,td,j}$ are the after-tax returns of fully taxable and tax-deferred investors, respectively.  

The return on fully taxable accounts is given by:

\begin{equation}
s_{e,ft,j} = (1-m_{j})E(1-\tau_{div}) + g_{j},
\end{equation}

\noindent\noindent where $m_{j}$ are the fraction of earnings that are retained by entity of type j, $\tau_{div}$ is the dividend tax rate on the marginal equity investor, and $g_{j}$ is the real return paid on retained earnings after the capital gains tax on the marginal equity investor.\footnote{If one subscribes to the ``new view", so that dividend taxes do not affect investment incentives, then the first term in this equation would be zero.  We use the subscript $j$ by the parameters $m$ and $g$, but note that these parameters only apply to business entities who can retain earnings (typically, these are those with an entity level tax).}
 
 The return on tax deferred accounts is:
 
 \begin{equation}
s_{e,td,j} = \frac{1}{Y_{td,j}}ln \left[(1-\tau_{td})e^{(\pi+E_{j})Y_{td,j}}+\tau_{td}\right]-\pi
\end{equation}
 

\subsection{Computing After-Tax Capital Gains}

Capital gains are not taxed until those gains are realized through the sale of stock.  The ability to defer the tax liability from gains complicates the calculation of the after-tax gains that accrue to investors. Further complicating this calculation is that, under current law, short and long term gains are taxed at differential rates and the basis for capital gains is ``stepped-up" on equity passed along to decedents upon death.  Note, we'll omit the tax entity type subscript here since this calculation only applies to those entity types that can retain earnings, namely C-corporations under current tax law. 

\begin{equation}
g = \omega_{scg}\times g_{scg} + \omega_{lcg}\times g_{lcg} + \omega_{xcg}\times mE,
\end{equation}
 
 \noindent\noindent wehre $\omega_{scg}$, $\omega_{lcg}$, and $\omega_{xcg}$ are the fractions of capital gains that are held for less than one year, more than one year but not until the owner's death, and those held until death, respectively.  The variables $g_{scg}$ and $g_{lcg}$ are the after-tax, real, annualized returns to short and long term capital gains.
 
\begin{equation}
g_{scg} = \frac{1}{Y_{scg}}\times ln\left[(1-\tau_{scg})e^{(\pi+mE)Y_{scg}}+\tau_{scg}\right]+\pi
\end{equation}

\noindent\noindent and

\begin{equation}
g_{lcg} = \frac{1}{Y_{lcg}}\times ln\left[(1-\tau_{lcg})e^{(\pi+mE)Y_{lcg}}+\tau_{lcg}\right]+\pi
\end{equation}


\section{Computing Fixed Assets by Industry and Entity Type}

In the computation of $\rho_{m,j}$, we need to have a measure of fixed assets by industry and tax treatment for each asset type, $\widetilde{FA}_{i,m,j}$.  To make this calculation, we work with two different sources of data.  The first is the BEA's \href{http://www.bea.gov/national/FA2004/Details/Index.html}{Detailed Data for Fixed Assets and Consumer Durable Goods}.  These data allow us to identify the stock of fixed assets by industry for each asset type.  Call this variable $FA_{i,m}$.  The second source of data we draw upon are the IRS Statistics of Income (SOI) data from business entity tax returns.  From these data, we use information on depreciable assets and accumulated depreciation, aggregated by industry and tax entity type to compute a measure of the total stock of fixed assets by industry and tax treatment, $FA^{\tau}_{m,j}$.  The superscript $\tau$ is used to denote that these asset values come from tax data.  Measuring assets from tax returns is not ideal for two reason.  First, there are reporting issues.  These line items do not affect tax liability and so are often not reported with as much accuracy as items related to income.  Relatedly, balance sheet reporting is often limited to businesses above a certain size.  The second reason is that, for the previously cited and other reasons, measures of asset from tax returns may not line up with BEA totals.  We thus use the asset totals computed from tax returns only to help apportion the BEA asset totals across tax treatment.  Namely, we compute $\widetilde{FA}_{i,m,j}$ as follows:

\begin{equation}
\label{eqn:asset_bridge}
\widetilde{FA}_{i,m,j} = FA_{i,m}\times \frac{FA^{\tau}_{m,j}}{\sum_{j=1}^{J} FA^{\tau}_{m,j}}
\end{equation}

\noindent\noindent This calculation makes the implicit assumption that the mix of asset types (i.e., the percent of total assets that each asset $i$ comprises) is the same across different tax entities within an industry.  

We define the set of tax entity types to be the following five entity types:
\begin{enumerate}
\item C-corporations
\item S-corporations
\item Corporate partners
\item Non-corporate partners
\item Sole proprietorships
\end{enumerate}

Investments by C-corporations and corporate partners face the corporate income tax treatment and thus will be defined a ``corporate".  Investments by other entity types face the individual income tax and will thus be defined as ``non-corporate".

There are several issues one faces when making the computation in Equation \ref{eqn:asset_bridge}.  The first set of issues has to do with varying specificity of industry classifications between the BEA and SOI data and within the SOI data.  The second set of issues relate to the reporting of entity types in the SOI data.  We discuss each in turn below.

\subsection{Handling Varying Industry Specificity}

The BEA data in the detailed fixed asset tables are the only source of data on asset types by industry.  We thus use the level industry specificity in those data as our baseline specificity and outline how we handle cases where the SOI data have more or less industry detail.  

In addition, the denominator of Equation \ref{eqn:asset_bridge} necessitates the summation of assets across different tax treatment types. Each of these tax treatment types draw upon data from different SOI tabulations.  These various tabulation have different degrees of industry specificity.  We thus first outline how we make consistent the tabulation across tax treatment types.

\subsubsection{Consistency Across SOI Data Sources}

SOI data come with two level of specificity.  Data from C corporations are available at what the SOI call ``minor industry" level.  This approximate encompass 196 industry classifications.  Data for other entity types are available at the ``major" industry level.\footnote{The exception to this are the data for S corporations, which we discuss in Section \ref{sec:CandS}.}  These approximate the 3-digit NAICS codes and encompass 81 industry classifications.\footnote{Pages 2-6 of the \href{https://www.irs.gov/pub/irs-soi/13cosbsec1.pdf}{Corporation Source Book} outline these industry classifications.}  To make these SOI data consistent, we aggregate the minor industries up to the major industry level.

Once we have all the SOI data aggregated to the SOI major industry code, we utilize a cross-walk to apply NAICS codes to the SOI data.

\subsubsection{SOI Data with More Industry Specificity than BEA Data}

If the SOI data have more specific industry groupings than the BEA data (for a specific BEA industry code $m$), we simply aggregate the SOI data ``up" to the level of industry detail provided by the BEA data.  In particular, we find $\widetilde{FA}_{i,m,j} $ as:

\begin{equation}
\widetilde{FA}_{i,m,j} = FA_{i,m}\times \frac{\sum_{m2\in m}FA^{\tau}_{m2,j}}{\sum_{j=1}^{J} \sum_{m2\in m} FA^{\tau}_{m2,j}}
\end{equation}

The subscript $m2$ refers to the more detailed industry classification in the SOI data.  

\subsubsection{SOI Data with Less Industry Specificity than BEA Data}

If the SOI data have less specific industry groupings than the BEA data (for a specific BEA industry code $m$), then we assume that the split of assets across tax treatment is the same for the ``children" (i.e. more minor industry) is the same as that of the ``parent (i.e., the more major industry).  In particular,

\begin{equation}
\widetilde{FA}_{i,m,j} = FA_{i,m}\times \frac{FA^{\tau}_{m3,j}}{\sum_{j=1}^{J} FA^{\tau}_{m3,j}}, \text{ where } m\in m3
\end{equation}

Here, $m3$ represents the less specific industry code from the SOI data.

\subsection{Dealing with SOI Reporting by Entity Type}

\subsubsection{C and S Corporation Data}
\label{sec:CandS}

Tax data on subchapter C corporations come from the data files for the \emph{SOI Tax Stats - Corporation Source Book} for 2011.  The link to those files is \href{http://www.irs.gov/uac/SOI-Tax-Stats-Corporation-Source-Book:-Data-File}{here}.  Specifically, we use the 2011sb1.csv and 2011sb3.csv files to find the aggregate amounts by industry for the following variables from Form 1120 and associated schedules: depreciable assets and accumulated depreciation.  Note that the 2011sb1.csv file contains data from all Form 1120 returns (which includes both C and S corporations).  Thus, in calculating aggregates for subchapter C corporations only, we net out the totals by industry and line item for S corporations using the 2011sb3.csv data.

Note that the level of industry detail in 2011sb1.csv and 2011sb3.csv differ, with the former reporting variables as fine as the 6-digit NAICS level and the latter reporting variables at the 2-digit level.  In order to infer S corporation data at a finer level of industry detail, we make the assumption that the each variable is distributed across minor industries within a major industry in the same way for all corporations as it they are for S corporations.  Letting $x_{m1}$ be a variable of interest reported for all corporations in detailed industry $m1$ (e.g., detailed may be a 6-digit NAICS code) from 2011sb1.csv and $x_{m2}$ be the same variable reported for all corporations at the less detailed industry level (e.g., 2-digit NAICS).  We thus assume that the variable $x$ (which could be depreciable assets or accumulated depreciation) for S corporations can be allocated across detailed industry categories $m1$ as:

\begin{equation}
x_{m1,s}=\frac{x_{m1}}{x_{m2}}\times x_{m2,s},
\end{equation}

\noindent\noindent where $m1\in m2$.  Variables allocated in this way are then used when differencing out data from 2011sb1.csv and 2011sb3.csv to find the amounts for C corporations.

Using these data, we calculate the stock of fixed assets for C corporations in industry $m$ as reported on tax returns as: ${FA}^{\tau}_{m,c}$ as the difference between the aggregate amounts of depreciable assets and accumulated depreciation for that industry.  

Tax data on subchapter S corporations come from the data files for the \emph{SOI Tax Stats - Corporation Source Book} for 2011.  The link to those files is \href{http://www.irs.gov/uac/SOI-Tax-Stats-Corporation-Source-Book:-Data-File}{here}.  Specifically, we use the 2011sb3.csv file to find the aggregate amounts by industry for the following variables from Form 1120S and associated schedules: depreciable assets, accumulated depreciation, land, beginning-of-year inventories, interest paid, capital stock, additional paid-in capital, retained earnings (appropriated and unappropriated), and cost of treasury stock.  

Using these data, we calculate the stock of fixed assets for S corporations in industry $m$ as reported on tax returns as: ${FA}^{\tau}_{m,s}$ as the difference between the aggregate amounts of depreciable assets and accumulated depreciation for that industry. 

\subsubsection{Partnership Data}

For partnerships, we draw upon the \href{http://www.irs.gov/uac/SOI-Tax-Stats-Partnership-Statistics-by-Sector-or-Industry}{SOI Tax Stats - Partnership Statistics by Sector or Industry}.  There are three files we use to get measures of partnership assets in 2012.  From the 12pa01.xls file, we pull aggregate depreciation deductions by industry.  From 12pa03.xls, we collect aggregate values for depreciable assets and accumulated depreciation.  Finally, we use 12pa05.xls to help us allocate the total partnership capital stock between corporate, individual, and tax exempt partners (we discuss this further below).

Using these data, we calculate the stock of fixed assets for partnerships in industry $m$ as reported on tax returns as: ${FA}^{\tau}_{m,p}$ as the difference between the aggregate amounts of depreciable assets and accumulated depreciation for that industry.  

\ \\
NOTE THAT FOR THESE DATA, INDUSTRY CODES ARE NOT GIVEN, SO WE'LL NEED TO MAP THE NAMES TO NAICS CODES.
\ \\
 
\textbf{\emph{Allocating Partnership Capital Across Types of Partners}}: Partners in partnerships may be corporations, individuals, partnerships, tax-exempts, or other organizations.  Because these partners face different tax treatment, we need to allocate shares of partnership assets to each of these entity types.  We do this by using ratios of depreciable assets to net income/loss by industry.  We then use these ratios to distribute the share of total assets across partner type using the net income/loss going to partners of a given type in each industry.  The assumption is that the ratio of assets to income/loss is the same across types of partners within a given industry.  This certainly misses some of the variation in the ownership structure of partnership assets and in the distribution of partnership income, but is a method that allows us to attribute partnership assets across partner types.

File 12pa03.xls provides data on depreciable assets by net income/loss by industry.  These data give totals for all partnerships and then totals separately for partnerships with net profits.  Denote these by $FA^{\tau}_{m,p,loss}$ and $FA^{\tau}_{m,p,gain}$ for partnerships with losses and gains, respectively.  We determine the total from partnerships with net losses as the difference between the total for all partnerships and the total from partnerships with positive net income.  We then calculate the ratios $\frac{FA^{\tau}_{m,p,loss}}{\text{Net Loss}_{m,p}}$ and $\frac{FA^{\tau}_{m,p,gain}}{\text{Net Gain}_{m,p}}$, for each industry $m$.  File 12pa01.xls gives the Net Income and Net Loss by industry that is used in the denominator of this calculation.  Using 12pa05.xls, we gather the aggregate amounts of net income or losses distributed to partners by partner type $t$ and industry $m2$, $\text{Net Income(Loss)}_{m2,t}$.  Note that the data from 12pa05.xls differ in the level of industry detail from the data in 12pa01.xls and 12pa03.xls.  For notational clarity, let $m1$ be the more detailed classifications and $m2$ be the less detailed classifications.  Using these two pieces of information together, we find the total amount of fixed assets by industry and partner type as:  

\begin{equation}
\text{FA}^{\tau}_{m2,p,t}=  \frac{\text{FA}^{\tau}_{m1,p,gain(loss)}}{\text{Net Income(Loss)}_{m1}} \times \text{Net Income(Loss)}_{m2,t},
\end{equation}

\noindent\noindent where $m1\in m2$ and $t$ denotes partner type (individual, corporate, partnership, tax-exempt, other).  To identify the amount of fixed assets by more detailed industry classifications, we make the assumption that the distribution of assets by partner type is the same for all minor industries within a major industry:

\begin{equation}
\text{FA}^{\tau}_{m1,p,t}=  \frac{\text{FA}^{\tau}_{m2,p,t}}{\text{FA}^{\tau}_{m2,p}} \times \text{FA}^{\tau}_{m1}
\end{equation}

When allocating capital across tax treatment, we will attribute the capital owned by corporate partners to the corporate sector and the remainder to the non-corporate sector.  But we can also keep corporate partners as a separate entity type in case one wants to look at them specifically.



\subsubsection{Sole Proprietorships}

We divide sole proprietorships into two groups: non-farm sole proprietors, who file a Schedule C of Form 1040, and farm sole proprietorships, who file Schedule F of Form 1040.  

\textbf{\emph{Non-farm Sole Proprietorships}}:  Our data for non-farm sole proprietorships come from the \href{http://www.irs.gov/uac/SOI-Tax-Stats-Nonfarm-Sole-Proprietorship-Statistics}{SOI Tax Stats - Non-farm Sole Proprietorship Statistics} for 2011.  Specifically, we use the file 11sp01br.xls.  These data do not record the value of depreciable assets for sole proprietorships, but they do contain depreciation deductions for sole proprietors.  Thus we impute the value of depreciable assets and land using the assumption that the ratio of depreciable assets to depreciation deductions is the same within a particular industry for sole proprietorships and partnerships.  Specifically, we find the stock of fixed assets for sole proprietors to be: 

\begin{equation}
{FA}^{\tau}_{sp}=\frac{\text{Depreciable Assets}_{m,p}}{\text{Depreciation Deductions}_{m,p}}\times \text{Depreciation Deductions}_{m,sp},
\end{equation}

\noindent\noindent where $m$ denotes industry and the subscripts $p$ and $sp$ represent partnership and sole proprietorship, respectively.  

\ \\
NOTE THAT FOR THESE DATA, INDUSTRY CODES ARE NOT GIVEN, SO WE'LL NEED TO MAP THE NAMES TO NAICS CODES.
\ \\

\textbf{\emph{Farm Sole Proprietorships}}:  The SOI do not provide detailed data on farm sole proprietors.  Thus for these businesses, we use \href{http://www.agcensus.usda.gov/Publications/2012/Full_Report/Volume_1,_Chapter_1_US/st99_1_067_067.pdf}{Table 67 of the \emph{2012 Census of Agriculture}} (\emph{COA}).  The \emph{COA} reports the values of land and structures (together) and the value of machinery and equipment.  These values are reported separately by type of organization (e.g, sole proprietorship, partnership).  To find the value of depreciable assets that is comparable to those for non-farm sole proprietors as reported in tax data, we must adjust these data so that we have a separate accounting of land and structures.  We use tax data to help us to impute this decomposition.  

Let $R_{sp}$ be the value of land and structures held by sole proprietor farms in the \emph{COA} and let $Q_{sp}$ be the value of machinery and equipment held by sole proprietor farms in the \emph{COA}.  Let $R_{p}$ and $Q_{p}$ be the analogous values for farm partnerships in the \emph{COA}.  By an accounting identity, it must be the case that $R_{i}+Q_{i}={FA}_{i}+{LAND}_{i}$ for entity of type $i\in{sp,p}$.  We thus find the ratio of land to capital held by partnerships in the agriculture industry; $\frac{\text{LAND}^{\tau}_{ag,p}}{{\text{LAND}^{\tau}_{ag,p}}+{FA}^{\tau}_{ag,p}}$, where the subscript $ag$ denotes the industry used is agriculture and the subscript $p$ denotes partnership returns. Next, this ratio is multiplied by the value for land and structures, $R_{p}$, and machinery and equipment. $Q_{p}$ for partnerships in the \emph{COA}.  The result is an imputation for the value of land held by farm partnerships: 

\begin{equation}
\text{LAND}_{p}= \frac{\text{LAND}^{\tau}_{ag,p}}{\text{LAND}^{\tau}_{ag,p}+{FA}^{\tau}_{ag,p}}\times (R_{p}+Q_{p})
\end{equation}

To then get an imputation for the value of land held by farm sole proprietorships, we assume that the distribution in the value of land per acre is the same for farm sole proprietorships as it is for farm partnerships.  That is, $\frac{\text{LAND}_{p}}{A_{p}}=\frac{\text{LAND}_{sp}}{A_{sp}}$, where $A_{p}$ and $A_{sp}$ denote the acreage held by farm partnerships and farm sole proprietorships, as reported in the \emph{COA}.  We use this assumption to solve for ${LAND}_{sp}$, given our imputed value for ${LAND}_{p}$ and data on $A_{p}$ and ${A}_{sp}$.  

Finally, we solve for the imputed value of fixed assets held by farm sole proprietorships as: 

\begin{equation}
{FA}_{sp}=R_{sp}+Q_{sp}-\text{LAND}_{sp}
\end{equation}

We then add the values of ${FA}_{sp}$ to the value for fixed assets that we found for non-farm sole proprietorships in the agriculture industry, ${FA}^{\tau}_{ag,sp}$.



\section{A Starting Point}

Let's get some initial calculations by assuming that $r_{m,j}$ does not vary by industry of tax treatment.  E.g., lets set $r=0.04$ for all $m$ and $j$.  In this case, the only component of the $METR$ that varies by industry and tax treatment is the cost of capital, $\rho_{i,m,j}$ and the $METR$ can be computed using only the BEA and SOI data detailed above.  Further, we'll assume property taxes are zero.

To find the variation in the $METR$ across different assets (and different tax treatment) we begin by finding the cost of capital:

\begin{equation}
\rho_{i,j} = \frac{(r-\pi)+\delta_{i}}{1-u_{j}}(1-u_{j}z_{i})-\delta_{i}
\end{equation}

Note that in computing the cost of capital by asset type and tax treatment, we do not need the BEA detailed fixed asset data or the SOI data. It only requires the BEA estimated depreciation rates by asset type and the IRS tax depreciation schedules by asset type.

The $METR$ is then calculated as:

\begin{equation}
METR_{i,j} =  \frac{\rho_{i,j} - (r-\pi)}{\rho_{i,j}},
\end{equation}

To find the variation in the $METR$ across industries and tax treatment, we need to use the BEA and SOI data on fixed assets to allow for the computation of the weighted average by industry:

\begin{equation}
\rho_{m,j} = \frac{\sum_{i=1}^{I}\widetilde{FA}_{i,m,j}\rho_{i,j}}{\sum_{i=1}^{I}\widetilde{FA}_{i,m,j}} ,
\end{equation}

and for the $METR$:

\begin{equation}
METR_{m,j} =  \frac{\rho_{m,j} - (r-\pi)}{\rho_{m,j}},
\end{equation}

\subsection{Finding the METR at Different Levels of Industry Specificity}

Starting at the baseline level of industry specificity that is given in the BEA detailed fixed asset data, we move up and down the NAICS ``tree", using two assumptions.  First, we assume that if we want to go ``down" the tree, to more detailed industry classifications, we assume the the cost of capital for each subindustry of a major industry is the same that for the major industry.  i.e., 

\begin{equation} 
\rho_{m2,j} = \rho_{m,j}, \forall m2 \in m
\end{equation}

This is simplistic, but the specificity in our BEA data sets the limit on the information we have on the mix of assets by industry.

Going ``up" the tree, we simply use the weight average approach noted above, with an additional sum that is over the minor industries making up the more major industry:

\begin{equation}
\rho_{m3,j} = \frac{\sum_{m\in m3}\sum_{i=1}^{I}\widetilde{FA}_{i,m,j}\rho_{i,j}}{\sum_{m\in m3}\sum_{i=1}^{I}\widetilde{FA}_{i,m,j}} ,
\end{equation}


\bibliography{BTax_bib}

\end{document}
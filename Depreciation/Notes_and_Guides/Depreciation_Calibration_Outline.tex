	\documentclass[article,11pt,letterpaper,fleqn]{article}
\usepackage{graphicx,color}
\usepackage{array}
\usepackage{threeparttable}
\usepackage[format=hang,font=normalsize,labelfont=bf]{caption}
\usepackage{colortbl}
\usepackage{multirow}
\usepackage{geometry}
\usepackage{subfigure}
\geometry{letterpaper,tmargin=1in,bmargin=1in,lmargin=1.25in,rmargin=1.25in}
\usepackage{hyperref}
\hypersetup{colorlinks,%
citecolor=red,%
filecolor=red,%
linkcolor=red,%
urlcolor=blue,%
pdftex}
\usepackage{amsmath}
\usepackage{amssymb}
\usepackage{amsthm}
\usepackage{harvard}
\usepackage{setspace}
\usepackage{float,graphicx,color}
\usepackage{appendix}
\usepackage{longtable}
\newtheorem*{thm}{Theorem}
\theoremstyle{definition}
\usepackage{lscape}
\numberwithin{equation}{section}
\newcommand{\cn}{\citeasnoun} % shortens command to cite as noun
\newcommand\ve{\varepsilon}


\title{Calibration of Firm Depreciation Parameters in the OLG Dynamic Scoring Model}
\date{\today}



% make tables with smaller sized font 
\makeatletter
\def\table{\@ifnextchar[{\table@i}{\table@i[\fps@table]}}
\def\table@i[#1]{\@float{table}[#1]\footnotesize}
\makeatother



%\setlength{\topmargin}{-0.4in}
%\setlength{\topskip}{0.3in}    % between header and text
%\setlength{\textheight}{9.0in} % height of main text
%\setlength{\textwidth}{6in}    % width of text
%\setlength{\oddsidemargin}{39pt} %even side margin
%\setlength{\evensidemargin}{39pt} %odd side margin

\begin{document}
\bibliographystyle{aer}
\maketitle



\begin{abstract}
This note outlines the process to calibrate the parameters of the firms' capital depreciation (both physical and tax depreciation rates).
\end{abstract}

\section*{General Idea for Calibrating Deprecation Parameters}

Our model will account for both physical (also called economic) depreciation rates and allowable deprecation rates as specified under tax law.  The differences between these two rates is an important element of tax policy/  For example., U.S. firms currently benefit from the fact that depreciation deductions for income tax purposes are generally accelerated as compared to the actual economic depreciation.  Thus the net present value of the deprecation deductions for tax purposes exceeds the net present value of the costs the firms incur resulting from economic deprecation of their assets.  

Note that depreciation rates (both economics and tax) differ by asset type.  For example, automobiles deprecate faster than do residential structures.  Due to computational concerns, our model will not capture the richness of multiple production industries and multiple types of capital.  Instead, we will consider multiple production industries, but only a single type of capital.  However, our calibration of depreciation rates will account for the fact that different industries and sectors (corporate vs. non-corporate) hold different types of capital in different amounts.  

``Capital type" is not well defined since the Bureau of Economic Analysis (BEA) (who produces the National Accounts numbers and provides estimates of economic deprecation rates as part of that) uses different categories of capital than does the IRS.  Probably the best thing to do (and what economists at Treasury and the Joint Committee on Taxation (JCT) do) is to look at the BEA classification of asset types (which is very detailed) and try to map that to the various asset lives used for tax policy.  This should be detailed enough for use when calculating economic depreciation and will be very useful when calculating the tax deprecation rates by sector and industry.  We'll thus create a ``transition matrix" to crosswalk between BEA and IRS asset types.  We'll discuss the method for doing this below, but what's important to understand at this point is that we'll be collecting data for both tax and economic deprecation rates  by production industry and sector.

To get the economic deprecation rate by industry and sector, we'll take a weighted average.  Assume there are $I$ types of capital.  We use the depreciation rate for each of those $I$ types find the weighted average where the weights are determined by the amount of capital of each type.  Thus the economic depreciation rate for capital in sector $C$ in industry $m$ can be give by:

\begin{equation}
\label{eqn:econ_deprec}
\delta^{C}_{m}=\sum_{i=1}^{I}\delta_{i}\frac{K^{C}_{i,m}}{K^{C}_{m}},
\end{equation}

\noindent\noindent where $K^{C}_{i,m}$ is the amount of capital of type $i$ in sector $C$ in industry $m$ and $K^{C}_{m}$ is the total amount of capital in sector $C$ in industry $m$.  Economic depreciation rates, $\delta_{i}$ will be found through the BEA's estimated depreciation rates by asset type.  The tax rate of depreciations will be calculated analogously: 

\begin{equation}
\label{eqn:tax_deprec}
\delta^{\tau C}_{m}=\sum_{i=1}^{I}\delta^{\tau C}_{i}\frac{K^{C}_{i,m}}{K^{C}_{m}},
\end{equation}

\noindent\noindent where $\delta^{\tau C}_{i}$ is the tax depreciation rate  of capital of type $i$ and is given by tax law.

Thus, there are three pieces of data we need to gather: 1) BEA estimates of capital stock by asset type, production industry, and sector (corporate vs. non-corporate), 2) BEA economic deprecation rates by BEA asset type,  and 3) Tax depreciation rates by IRS asset type.  The steps below will help to guide you through obtaining each of these pieces of data.  Finally, I provide instructions for the transition matrix used to map BEA asset types into IRS asset types and creating the depreciation rates by industry and sector that will be input into the computational  model.

\section*{Step 1: Acquire the BEA Meaures of Capital Stock}
\label{sec:step3}

We are going to want to account for differences in the physical deprecation of capital across different production industries and sectors.  To do this, we'll use a weighted average calculation, which means we'll need data on capital stock by asset type across industry-sector splits.  Table \ref{tab:prod_ind} summarizes the production industries we want to consider.\footnote{This excludes the multi-national sector, which we still need to think about.}  These generally correspond to two-digit NAICS codes, with some exceptions for industries that have (or may have) special tax treatment.

% Table generated by Excel2LaTeX from sheet 'ProductionIndustries'
\begin{table}[htbp]
  \centering
  \caption{Production Industries}
    \begin{tabular}{lll}
    \hline
    \hline
    \# & NAICS Code & Industry \\
    \hline
    1     & 11    & Agriculture, Forestry, Fishing and Hunting \\
    2     & 211   & Oil and Gas Extraction \\
    3     & 212 and 213 & Mining and Support Activities for Mining \\
    4     & 22    & Utilities \\
    5     & 23    & Construction \\
    6     & 32411 & Petroleum Refineries \\
    7     & 336   & Transportation Equipment Manufacturing \\
    8     & 3391  & Medical Equipment and Supplies Manufacturing \\
    9     & Other codes in 31-33 & Manufacturing \\
    10    & 42    & Wholesale Trade \\
    11    & 44-45 & Retail Trade \\
    12    & 48-49 & Transportation and Warehousing \\
    13    & 51    & Information \\
    14    & 52    & Finance and Insurance \\
    15    & 53    & Real Estate and Rental and Leasing \\
    16    & 54    & Professional, Scientific, and Technical Services \\
    17    & 55    & Management of Companies and Enterprises \\
    18    & 56    & Administrative and Support and Waste Management and Remediation Services \\
    19    & 61    & Educational Services \\
    20    & 62    & Health Care and Social Assistance \\
    21    & 71    & Arts, Entertainment, and Recreation \\
    22    & 72    & Accommodation and Food Services \\
    23    & 81    & Other Services (except Public Administration) \\
    24    & 92    & Public Administration \\
      \hline
    \hline
    \end{tabular}%
  \label{tab:prod_ind}%
\end{table}%

A place to start looking for capital stock data my production industry is here: \\
\href{http://www.bea.gov/national/FA2004/Details/Index.html}{http://www.bea.gov/national/FA2004/Details/Index.html}

Another source is: \href{http://www.bea.gov/iTable/index\_FA.cfm}{http://www.bea.gov/iTable/index\_FA.cfm}

Note that we can just use the most recent year's data, so ``current cost" valuation should be fine (so long as all our capital stock data come from the same year).

Spend some time comparing sources and evaluate whether one source offers more industry-sector-asset type detail than another.

Note that I've had a hard time finding capital stocks by industry-sector (I can find by sector and by industry, but not both).  Try to find this, but if we can't we can just assume that the fraction of each type of capital held is the same across the corporate and non-corporate sectors.


\section*{Step 2: Acquire the BEA Depreciation Rates}
\label{sec:step1}

We are going to want to account for differences in the physical deprecation of capital across different types of capital.  The BEA produces such estimates.  One source is: \href{http://bea.gov/national/FA2004/Tablecandtext.pdf}{http://bea.gov/national/FA2004/Tablecandtext.pdf}

Note that it may take some effort to map the asset types here into those from the data in Step 1.  For mapping we need to do manually (i.e., we aren't just matching codes that are consistent from one data source to another), we need to produce a crosswalk file describing exactly how things were mapped.  I'll be glad to help you create these.

\section*{Step 3: Acquire the IRS Depreciation Rates}
\label{sec:step3}

The list of asset types considered by the IRS is much smaller than that in the detailed data on economic depreciation rates provided by the BEA.  To see these asset categories, you can reference the instructions for Form 4562 (the form where tax filers describe their depreciation deductions): \href{http://www.irs.gov/pub/irs-pdf/i4562.pdf}{http://www.irs.gov/pub/irs-pdf/i4562.pdf}

Note that the IRS's deprecation schedule is not a ``straight-line" schedule - the percentage of depreciation varies each year.  Our computational model cannot account for different aged capital, so what we need to do is convert these deprecation schedule for each asset type into a straight-line-equivalent schedule.  We will do this by finding the straight-line deprecation amount (percentage depreciation rate by year - constant over the life of the asset) that gives the same net present value as the IRS's accelerated deprecation schedule.

Note that in the future we may need to revisit these deprecation rates since not all depreciation deductions are taken (e.g., the firm may have not further tax liability offset with a deduction) and so we might need to make some adjustments. 


\section*{Step 4: Create Crosswalk Between BEA and IRS Asset Types}
\label{sec:step4}

There are two options to make this mapping: 1) Find the economic life of the asset as determined by the BEA and then assign that a tax type that has a similar (but shorter, since tax depreciation is accelerated) depreciable life. or 2) Use the asset type descriptions from BEA and then map that into the assets descriptions for the various asset types used in tax (see the Form 4562 instructions: \url{http://www.irs.gov/pub/irs-pdf/i4562.pdf}).  We also have to be careful that we don't just consider assets of the type of Form 4562 (\url{http://www.irs.gov/pub/irs-pdf/f4562.pdf}), but also consider assets that received immediate expensing for tax purposes (like intangibles) and inventories.

\section*{Step 5: Calculate Economic Depreciation Rates by Industry and Sector}
\label{sec:step5}

To get the economic deprecation rate by industry and sector, we'll take a weighted average.  Assume there are $I$ types of capital.  We use the depreciation rate for each of those $I$ types find the weighted average where the weights are determined by the amount of capital of each type.  Thus the economic depreciation rate for capital in sector $C$ in industry $m$ can be give by:

\begin{equation}
\label{eqn:econ_deprec}
\delta^{C}_{m}=\sum_{i=1}^{I}\delta_{i}\frac{K^{C}_{i,m}}{K^{C}_{m}},
\end{equation}

\noindent\noindent where $K^{C}_{i,m}$ is the amount of capital of type $i$ in sector $C$ in industry $m$ and $K^{C}_{m}$ is the total amount of capital in sector $C$ in industry $m$.  Economic depreciation rates, $\delta_{i}$ will be found through the BEA's estimated depreciation rates by asset type.

\section*{Step 6: Calculate Tax Depreciation Rates by Industry and Sector}
\label{sec:step6}

 To get the  tax rate of depreciation by industry and sector: 

\begin{equation}
\label{eqn:tax_deprec}
\delta^{\tau C}_{m}=\sum_{i=1}^{I}\delta^{\tau C}_{i}\frac{K^{C}_{i,m}}{K^{C}_{m}},
\end{equation}

\noindent\noindent where $\delta^{\tau C}_{i}$ is the tax depreciation rate  of capital of type $i$ and is given by tax law.

\section*{Notes}
\label{sec:notes}

We can probably start by handling these data in Excel.  Note that we may want to eventually write Python scripts to read in an manipulate the data, but let's start with software that I am more familiar with (so Stata or Excel would be ideal).  We should be careful to note exactly where these files were obtained from and write our scripts such that they read in files from these sources.

I have not done this before and there are sure to be issues - let me know whenever you have questions.  Note that we may have trouble finding data at certain levels of industry-sector-asset type detail.  We should exhaustively search for the data, but if it is unavailable we can always aggregate things up a bit.

I'm always happy to answer any questions (\href{mailto:jason.debacker@gmail.com}{jason.debacker@gmail.com}, 770-289-0340).  

\bibliography{TaxModelCalibrationReferences}

\end{document}
